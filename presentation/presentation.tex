%%%%%%%%%%%%%%%%%%%%%%%%%%%%%%%%%%%%%%%%%
% Beamer Presentation
% LaTeX Template
% Version 1.0 (10/11/12)
%
% This template has been downloaded from:
% http://www.LaTeXTemplates.com
%
% License:
% CC BY-NC-SA 3.0 (http://creativecommons.org/licenses/by-nc-sa/3.0/)
%
%%%%%%%%%%%%%%%%%%%%%%%%%%%%%%%%%%%%%%%%%

%----------------------------------------------------------------------------------------
%	PACKAGES AND THEMES
%----------------------------------------------------------------------------------------

\documentclass[12pt]{beamer}

\mode<presentation>{

	% The Beamer class comes with a number of default slide themes
	% which change the colors and layouts of slides. Below this is a list
	% of all the themes, uncomment each in turn to see what they look like.

	%\usetheme{default}
	%\usetheme{AnnArbor}
	%\usetheme{Antibes}
	%\usetheme{Bergen}
	%\usetheme{Berkeley}
	%\usetheme{Berlin}
	%\usetheme{Boadilla}
	%\usetheme{CambridgeUS}
	%\usetheme{Copenhagen}
	%\usetheme{Darmstadt}
	%\usetheme{Dresden}
	%\usetheme{Frankfurt}
	%\usetheme{Goettingen}
	%\usetheme{Hannover}
	%\usetheme{Ilmenau}
	%\usetheme{JuanLesPins}
	%\usetheme{Luebeck}
	\usetheme{Madrid}
	%\usetheme{Malmoe}
	%\usetheme{Marburg}
	%\usetheme{Montpellier}
	%\usetheme{PaloAlto}
	%\usetheme{Pittsburgh}
	%\usetheme{Rochester}
	%\usetheme{Singapore}
	%\usetheme{Szeged}
	%\usetheme{Warsaw}

	% As well as themes, the Beamer class has a number of color themes
	% for any slide theme. Uncomment each of these in turn to see how it
	% changes the colors of your current slide theme.

	%\usecolortheme{albatross}
	%\usecolortheme{beaver}
	%\usecolortheme{beetle}
	%\usecolortheme{crane}
	%\usecolortheme{dolphin}
	%\usecolortheme{dove}
	%\usecolortheme{fly}
	%\usecolortheme{lily}
	%\usecolortheme{orchid}
	%\usecolortheme{rose}
	%\usecolortheme{seagull}
	\usecolortheme{seahorse}
	%\usecolortheme{whale}
	%\usecolortheme{wolverine}

	%\setbeamertemplate{footline} % To remove the footer line in all slides uncomment this line
	%\setbeamertemplate{footline}[page number] % To replace the footer line in all slides with a simple slide count uncomment this line

	%\setbeamertemplate{navigation symbols}{} % To remove the navigation symbols from the bottom of all slides uncomment this line
}

\usepackage{graphicx} % Allows including images
\usepackage{booktabs} % Allows the use of \toprule, \midrule and \bottomrule in tables

\usepackage[utf8]{inputenc}
\usepackage[ngerman]{babel}

\usepackage{ulem}
\usepackage{tikz}
\usetikzlibrary{arrows,shapes,positioning}


% lstlisting package

%==IMAGEPATH============================
\graphicspath{ {images/} }
%----------------------------------------------------
%==NEWCOMMAND=FOR=IMAGES============================
%img{\$name}{\$caption}{\$label}{\$witdh}
\newcommand{\img}[4] {
  \begin{figure}[H]
    \centering
    \includegraphics[width=#4\textwidth]{#1}
    \caption{#2}
    \label{fig:#3}
  \end{figure}}
%----------------------------------------------------

\usepackage{stmaryrd}
\usepackage{multirow}
\usepackage{amsmath}
\usepackage{listings}
\usepackage{lstautogobble}
\usepackage{wrapfig}
\definecolor{dkgreen}{rgb}{0,0.6,0}
\definecolor{gray}{rgb}{0.5,0.5,0.5}
\definecolor{mauve}{rgb}{0.58,0,0.82}
\lstset{
	%numbers = left,
	frame = single,
	basicstyle=\footnotesize, % the size of the fonts that are used for the code
	breakatwhitespace=false, % sets if automatic breaks should only happen at whitespace
	breaklines=true, % sets automatic line breaking
	captionpos=b, % sets the caption-position to bottom
	commentstyle=\color{dkgreen}, % comment style
	keywordstyle=\color{blue}, % keyword style
	%    language=C++, % the language of the code
	mathescape=true,
	basicstyle=\ttfamily\scriptsize,
autogobble}
\lstdefinelanguage{diff}{
	morecomment=[f][\color{blue}]{@@},     % group identifier
	morecomment=[f][\color{red}]-,         % deleted lines 
	morecomment=[f][\color{green}]+,       % added lines
	morecomment=[f][\color{magenta}]{---}, % Diff header lines (must appear after +,-)
	morecomment=[f][\color{magenta}]{+++},
}
%----------------------------------------------------------------------------------------
%	TITLE PAGE
%----------------------------------------------------------------------------------------

\title[Machine Learning -- ILP]{Inductive Logic Programming} % The short title appears at the bottom of every slide, the full title is only on the title page

\author{Christian Bay} % Your name
\institute[FAU] % Your institution as it will appear on the bottom of every slide, may be shorthand to save space
{
	FAU Erlangen-Nürnberg \\ % Your institution for the title page
	\medskip
	\textit{christian.bay@fau.de} % Your email address
}
\date{04.02.2016} % Date, can be changed to a custom date

\begin{document}

% Print Titlepage
\begin{frame}
	\titlepage
\end{frame}
\tableofcontents

\section{Einführung}

\frame{\frametitle{Agenda}\tableofcontents[currentsection]}
\begin{frame}
	\frametitle{Induction Logic Programming -- ILP}

	\begin{block}{Was ist ILP}
		\begin{itemize}
			\item Die Schnittstelle zwischen Machine Learning und logischer Programmierung
			\item {Idee: Formalisierung der \textit{Umgebung}
				\begin{itemize}
					\item Basis: Background-Knowledge, positive und negative Beispiele
					\item Darstellung in Form von Klauseln
					\item Generierung neuen Wissens durch das finden von Hypothesen die positive
					Klauseln erfüllen, negative jedoch nicht.
				\end{itemize}
			}
		\end{itemize}
	\end{block}
	Wieso induktiv?
	$\Rightarrow$ Lösen von Problemen vom spezifischen zum generellen
\end{frame}

\begin{frame}
	\frametitle{First order logic -- FOL}
	\begin{block}{}
	In \textbf{FOL} besteht die Welt aus
	% TODO: Begriffe nochmal genauer anschauen: Insbesondere Atome
	\begin{itemize}
		\item Objekten  Personen, Dingen)
		\item Prädikaten $(>, <)$
		\item Funktionen $(+, -)$
	\end{itemize}
	\end{block}


	Terme beschreiben Objekte in der Welt:\\
	Konstanten ($2$, Chuck Norris), Variablen (x,y, a, \ldots) und
	Funktionen von Termen sind Terme.

	\textit{Ground term}: Ein Term ohne Variablen.

	\begin{block}{}
		Atome/Literale sind kleinstmögliche Ausdrücke, die wahr/falsch sind
		\begin{itemize}
			\item $predicate(Term_1, \ldots, Term_n)$\\ 
				Beispiel: is\_human(Pinoccio)
			\item $Term_1 = Term_2$ (Gleicheitsrelation)
		\end{itemize}
	\end{block}

\end{frame}

\begin{frame}
	\frametitle{Art der Logik -- Beispiel I }

	\begin{block}{Syntaktische Beschreibung der Welt}
		$\text{Konstante} = \{\text{Zero}\}$\\
		$\text{Funktion} =\{Succ/1, Plus/2 \}$\\
		$\text{Prädikate} =\{even/1, is\_greater/2 \}$
	\end{block}

\end{frame}

\begin{frame}
	\frametitle{Art der Logik -- Beispiel II}
	Interpretation $\mathcal{I}$ besitzt eine Grundmenge $\mathcal{M}$
	und bildet Konstanten und Funktionen auf Elemente der Grundmenge ab.

	Prädikate werden auf Wahrheitswerte abgebildet

	\begin{block}{Interpretation}
			$\mathcal{M} = \mathbb{N}_0$\\
			$\mathcal{I}(Zero) = 0$\\
			$\mathcal{I}(Plus(Succ(Zero), Succ(Zero))) = 2$\\
			$\mathcal{I}(even(Suc(Suc(Zero)))) = True$
	\end{block}
\end{frame}

\begin{frame}
	\frametitle{Art der Logik -- Beispiel III}
	\begin{block}{Logische Folgerung}
		Man schreibt
		\begin{align*}
			 \bigwedge\Gamma \vDash \varphi
		\end{align*}
		Genau dann wenn jede Intepretation $\mathcal{I}$, die alle Formeln
		der Menge $\Gamma$ erfüllt, auch $\varphi$ erfüllt.
	\end{block}
\end{frame}

\begin{frame}
	\frametitle{First order logic und ILP}
	Horn Clauses: Klausel mit maximal einem negativen Literal.

	\begin{align*}
		\neg C_1 \vee \neg C_2 \vee \ldots \vee \neg C_n  \vee C_{n+1} \Leftrightarrow\\
		C_1 \wedge C_2 \wedge \ldots \wedge C_n  \rightarrow C_{n+1}
	\end{align*}

	ILP verwendet zumeist \textit{definite program clauses}(genau ein negatives Literal) vom Typ:

	\begin{align*}
		\underbrace{T}_{\text{head}} \Leftarrow \underbrace{L_1, \ldots, L_m}_{\text{body}}
	\end{align*}
	wobei $T, L_1, \ldots, L_m$ Literale sind, wobei $T$ immer positiv ist.

	Beispiel:
	\begin{align*}
		daughter(X,Y) \Leftarrow female(X), mother(Y, X)
	\end{align*}
\end{frame}


\begin{frame}
\frametitle{Hypothesen, Beispiele etc}
	\emph{Gegeben:} Background-Knowledge $(B)$, Menge von positiven $(E^+)$ und negativen $(E^-)$ Beispielen\\
	\emph{Ziel:} Finden einer Hypothese $(\mathcal{H})$ die alle positiven und kein negatives Beispiele erfüllt
\begin{figure}[H]
	\centering
	\includegraphics[width=0.5\textwidth]{hypothesis}
\end{figure}
\end{frame}

\begin{frame}
	\frametitle{Gewünschte Hypotheseneigenschaften}
	\begin{itemize}
		\item Necessity:          $B \nvDash E^{+}$

			Beispiele dürfen nicht durch Hintergrundwissen allein schon erklärbar sein

		\item Sufficiency:        $B \wedge H \vDash E^{+}$

			Hypothese $\mathcal{H}$ muss alle positiven Beispiele logisch folgern

		\item Weak consistency:   $B \wedge H \nvDash false$

			$H$ darf dem Hintergrundwissen nicht widersprechen

		\item Strong consistency: $B \wedge H \wedge E^{-} \nvDash false$

			$H$ darf negativen Beispielen nicht widersprechen
	\end{itemize}
\end{frame}

\section{Techniken des Inductive Logic Programming}
\frame{\frametitle{Agenda}\tableofcontents[currentsection]}
\subsection{Finden des Hypothesenraums}
\begin{frame}
	\frametitle{Subsumption von Literalen}
	\emph{Idee}: Eingrenzung des Suchraums erleichtert Hypothesensuche

	\begin{block}{Definition: Subsumption für Literale/Atome}
		Seien $L_1$ und $L_2$ Literale: $L_1 \underbrace{\preceq}_{\text{subsumiert}} L_2$
		wenn eine Substitution $\theta$ existert, sodass:
		\begin{align*}
			 L_1 \theta \subseteq L_2
		\end{align*}
	\end{block}
	\begin{bsp}
		\begin{align*}
			 daughter(X, Y)\theta\subseteq daughter(mary, ann)\\\text{  mit  } \theta = \{X/mary, Y/ann\}
		\end{align*}
	\end{bsp}
\end{frame}
\begin{frame}
	\frametitle{Subsumption von Klauseln}
	Das gleiche Prozedere für Klauseln:
	\begin{block}{Definition: Subsumption für Klauseln}
		Seien $C_1$ und $C_2$ Klauseln: $C_1 \underbrace{\preceq}_{\text{subsumiert}} C_2$
		wenn eine Substitution $\theta$ existert, sodass:
		\begin{align*}
			 C_1 \theta \subseteq C_2
		\end{align*}
	\end{block}
	\begin{bsp}
		\begin{gather*}
			(daughter(X, Y)\leftarrow parent(Y,X))\theta \subseteq\\
			 daughter(mary, ann) \leftarrow parent(ann, mary), parent(ann, tom)\\
			 \text{  mit  } \theta = \{X/mary, Y/ann\}
		\end{gather*}
	\end{bsp}
\end{frame}

\begin{frame}
	\frametitle{Eigenschaften der Subsumption}
	\begin{enumerate}
		\item {
			Wenn $C_1 \preceq C_2$ dann gilt (nicht umgekehrt!):
			\begin{align*}
				C_1 \vDash C_2
			\end{align*}
		}
		\item{ Die Relation $\preceq$ führt ein Gitter (lattice) von reduzierten Klauseln ein
			\begin{figure}[H]
				\begin{center}
					\begin{tikzpicture}
						\node (G) at (1.5,2) {$\top$};
						\node (A) at (1.5,1.5) {$p(X,Y)$};
						\node (B) at (-1,0.5) {$p(X,a)$};
						\node (C) at (1.5,0.5) {$p(X,X)$};
						\node (D) at (4,0.5) {$p(a,Y)$};
						\node (E) at (1.5,-0.5) {$p(a,a)$};
						\node (F) at (1.5,-1) {$\bot$};
			
						\path [-] (A) edge node[left] {} (G);
						\path [-] (A) edge node[left] {} (B);
						\path [-] (A) edge node[left] {} (C);
						\path [-] (A) edge node[left] {} (D);
						\path [-] (B) edge node[left] {} (E);
						\path [-] (C) edge node[left] {} (E);
						\path [-] (D) edge node[left] {} (E);
						\path [-] (F) edge node[left] {} (E);
					\end{tikzpicture}
				\end{center}
				\caption{Partiell geordnete Menge von Formeln (POSET)}
				\label{fig:poset_atomic}
			\end{figure}
		}
	\end{enumerate}
\end{frame}

\subsection{Bottom-up -- Least general generalization}
\begin{frame}
	\frametitle{Bottom-up -- Least general generalization (LGG)}
	 Die \textit{least general generalization} von zwei Klauseln $C_1, C_2$ ist
	 die kleinste obere Schranke von $C_1,C_2$ im Gitter.

	\begin{block}{Erkenntnisse}
			\begin{itemize}
				\item [$\Rightarrow$] Alle Beispiele die $C_1$ abdeckt werden auch von
				allen 'kleineren Klauseln' abgedeckt
				\item[$\Rightarrow$] Wenn $C_1$ ein Beispiel nicht abdeckt,
				dann tut dies auch keine 'kleinere Klausel'
			\end{itemize}
	 \end{block}
\end{frame}

\begin{frame}
	\frametitle{Bottom-up -- Least general generalization (LGG)}
	\begin{block}{$lgg$ von Termen}
		\begin{enumerate}
			\item $lgg(t,t) = t$\\
			\item $lgg(f(s_1, \ldots, s_n), f(t_1, \ldots, t_n)) = f(lgg(s_1, t_1), \ldots, lgg(s_n, t_n))$
			\item Sonst: $lgg(t_1, t_2) = v , v$ freie Variable
		\end{enumerate}
	\end{block}
	\begin{block}{$lgg$ von Literalen}
	\begin{enumerate}
		\item $lgg(p(u_1, \ldots, u_n), p(s_1, \ldots, s_n)) = p(lgg(u_1, s_1), \ldots, lgg(u_n, s_n))$\\
		\item Sonst: $lgg(L_1, L_2) = \top$
	\end{enumerate}
	\end{block}
\end{frame}


\begin{frame}
\frametitle{Bottom-up -- Relative least general generalization}
\begin{block}{Definition -- Relative least general generalization}
	Gegeben: Hintergrundwissen $\mathcal{B}$ besteht nur aus \emph{ground literals} (Literale ohne Variablen).
	Sei $K$ die Konjunktion aller Hintergrundliterale und $e_1, e_2$ je positive Beispiele
	\begin{align*}
		rlgg(e_1, e_2) = lgg((e_1 \leftarrow K), (e_2 \leftarrow K))
	\end{align*}
\end{block}

\end{frame}


\begin{frame}{Linearity}
\frametitle{Bottom-up -- Relative least general generalization}
	\pgfdeclarelayer{background}
	\pgfsetlayers{background,main}
	\tikzstyle{vertex}=[rectangle,fill=black!25,minimum size=20pt,inner sep=0pt]
	\tikzstyle{selected vertex} = [vertex, fill=red!24]
	\tikzstyle{edge} = [draw,thick,-]
	\tikzstyle{weight} = [font=\small]
	\tikzstyle{selected edge} = [draw,line width=5pt,-,red!50]
	\setbeamercovered{invisible}
	\begin{figure}
		\begin{tikzpicture}[scale=1.0, auto,swap]
		\node[vertex] (a) at (0,0) {$A = e_1 \leftarrow K$};
		\node[vertex] (b) at (2.5,0) {$B = e_2 \leftarrow K$};
		\node[vertex] (c) at (5,0) {$C = e_3 \leftarrow K$};
		\node[vertex] (d) at (7.5,0) {$D = e_4 \leftarrow K$};\pause
		\node[vertex] (e) at (1.25,2) {$C' = lgg(A, B)$};
		\node[vertex] (f) at (6.25,2) {$C''= lgg(C, D)$};\pause
		\node[vertex] (g) at (3.75,4) {$\mathcal{H} = lgg(C', C'')$};
		\begin{pgfonlayer}{background}
			\path<2->[selected edge] (a.center) -- (e.center);
			\path<2->[selected edge] (b.center) -- (e.center);
			\path<2->[selected edge] (c.center) -- (f.center);
			\path<2->[selected edge] (d.center) -- (f.center);
			\path<3->[selected edge] (e.center) -- (g.center);
			\path<3->[selected edge] (f.center) -- (g.center);
		\end{pgfonlayer}
		\end{tikzpicture}
	\end{figure}
\end{frame}

\begin{frame}
	\frametitle{Bottom-up -- Relative least general generalization}
	Beispiel: Familienkonstellation (siehe Terminal)
	\begin{figure}[H]
		\begin{center}
			\begin{tikzpicture}[scale=0.6]
				\node (A) at (0, 2)  {\color{blue}georg};
				\node (B) at (2,0)   {\color{red}mary};
				\node (C) at (4, 2)   {\color{red}ann};
				\node (D) at (6,0)   {\color{blue}tom};
				\node (E) at (8, -2) {\color{red}eve};

				\path [->] (A) edge node[above] {} (B);
				\path [<-] (B) edge node[above] {} (C);
				\path [->] (C) edge node[above] {} (D);
				\path [->] (D) edge node[above] {} (E);
			\end{tikzpicture}
		\end{center}
		\caption{Eine schrecklich nette Famlie}
	\end{figure}
\end{frame}

\begin{frame}
	\frametitle{Bottom-up -- Relative least general generalization}
	\begin{block}{Hintergrundwissen der Familienkonstellation}
		\begin{align*}
			K   &= p(a,m), p(a,t), p(t,e), p(t,i), f(a), f(m), f(e)\\
			e_1 &= daughter(m, a)\\
			e_2 &= daughter(e, t)
		\end{align*}
	\end{block}

	\begin{gather*}
		\begin{split}
			\emph{d(V_{m,e}, V_{a,t})} \leftarrow &\; p(a,m), p(a,t), p(t,e), p(t,i), f(a), f(m), f(e),\\
			&\; p(a, V_{m,t}), \emph{p(V_{a,t}, V_{m,e})}, p(V_{a,t}, V_{m,i}), p(V_{a,t}, V_{t,e}),\\
			&\; p(V_{a,t}, V_{t,i}) ,p(t, V_{e,i}), f(V_a, m), f(V_{a,e}), \emph{f(V_{m,e})}
		\end{split}
	\end{gather*}
\end{frame}

\begin{frame}
	\frametitle{Bottom-up -- Relative least general generalization}
	\begin{itemize}
		\item $K$ darf nur aus \textit{ground literals} bestehen (Anhang: Saturierung)
		\item $\mathcal{H}$ kann nicht mehr als \emph{eine} Klausel umfassen
		\item Kombination der $lgg$-Paare bestimmt Qualität der Hypothese
		\item \emph{Große} $\mathcal{H}$ bereits bei kleinen Beispielen

	\end{itemize}
\end{frame}


\subsection{Bottom-up -- Inverse Resolution}
\begin{frame}
	\frametitle{Bottom-up -- Inverse resolution [Muggleton '88]}
	Plotkins \textbf{LGG} Ansatz stark beschränkt

	Alternative Idee: \textbf{Inverse resolution}

	\begin{align*}
		B \wedge H &\vDash E \Leftrightarrow\\
		B \wedge \neg E &\vDash \neg H
	\end{align*}

	\vspace{5pt}

	Suchen einer \textit{Bridgefunktion} $F$:
	\vspace{5pt}
	\begin{align*}
		B \wedge \neg E     &\vDash F\\
		F                   &\vDash \neg H\\
		\Rightarrow \neg F  &\Dashv H\\
		\Rightarrow \neg F  &\succeq H\\
	\end{align*}
\end{frame}

\begin{frame}
	\frametitle{Bottom-up -- Inverse Resolution}
	Generelle Idee: Laufe den Herleitungsbaum rückwärts

	\begin{block}{Herleitungsbaum Aussagenlogik}
		Sei Theorie $\mathcal{T}=\{ u \leftarrow v; v \leftarrow w, w\}$ gegeben.
		Herleitung von $u$:
		\begin{figure}[H]
			\begin{center}
				\begin{tikzpicture}[scale=0.6]
					\node (A) at (0.3, 0) {$u$};
					\node (B) at (-3.6,2.15) {$u \leftarrow v$};
					\node (C) at (-2,4) {$v \leftarrow w$};
					\node (D) at (2,4) {$w$};
					\node (E) at (1.3, 2.2) {$v$};

					\path [->] (B) edge node[above] {} (A);
					\path [->] (C) edge node[above] {} (E);
					\path [->] (D) edge node[above] {} (E);
					\path [->] (E) edge node[above] {} (A);
				\end{tikzpicture}
			\end{center}
		\end{figure}
	\end{block}
\end{frame}

\begin{frame}
	\frametitle{Bottom-up -- Inverse Resolution}
	Herleitungen in \textit{First order logic} benötigt zusätzlich Substitutionen:

	\begin{itemize}
		\item $\mathcal{H} = \{c\} = \{daugther(X,Y) \leftarrow female(X), parent(Y,X)\}$
		\item $\mathcal{B} = \{b_1, b_2\}$ mit $b_1 = female(mary)$ und
			$b_2 = parent(ann, mary)$
		\item $\mathcal{T} = \mathcal{B} \cup \mathcal{H}$
		\begin{figure}[H]
			\begin{center}
				\begin{tikzpicture}[scale=0.6]
					\node (A) at (0.3, 0) {\tiny{$daugher(mary,ann)$}};
					\node (B) at (-3.6,2.15) {\tiny{$b_2 = parent(ann,mary)$}};
					\node (C) at (-2,4) {\tiny{$b_1 = female(mary)$}};
					\node (D) at (2.5,5) {\tiny{$daughter(X,Y) \leftarrow female(X), parent(Y,X)$}};
					\node (E) at (1.3, 2.2) {\hspace{1cm}\tiny{$daughter(mary, Y) \leftarrow parent(Y,mary)$}};
					\node (S1) at (2.3, 1.2) {\tiny{$\theta_2 = \{Y/ann\}$}};
					\node (S1) at (3, 3.6) {\hspace{0.5cm}\tiny{$\theta_1 = \{X/mary\}$}};


					\path [->] (B) edge node[above] {} (A);
					\path [->] (C) edge node[above] {} (E);
					\path [->] (D) edge node[above] {} (E);
					\path [->] (E) edge node[above] {} (A);
				\end{tikzpicture}
			\end{center}
		\end{figure}
	\end{itemize}
\end{frame}

\begin{frame}
	\frametitle{Bottom-up -- Inverse Resolution}
		\textbf{Inverse Substitution}: Umkehrung der Substitution:
		\begin{bsp}
			\begin{align*}
				c &= daughter(X, Y) \leftarrow female(X), parent(Y,X)\\
				\text{ mit } \theta &= \{X/mary, Y/ann\}\\
			\end{align*}
		\end{bsp}

		\begin{figure}[H]
			\begin{center}
				\begin{tikzpicture}[scale=0.6]
					\node[text width=3cm] (A) at (0, 0) {\tiny{$c' = daughter(mary, ann) \leftarrow female(mary),
					parent(ann, mary)$}};
					\node[text width=3cm] (B) at (8, 0) {\tiny{$c = daughter(X, Y)\leftarrow female(X), parent(Y,X)$}};
				    \path[dashed,->] (A) edge [red, bend left=70]  node[above] {$c'\theta$} (B);
					\path[dashed,->] (B) edge [blue, bend left=50]  node[above] {$c\theta^{-1}$} (A);
				\end{tikzpicture}
			\end{center}
		\end{figure}
\end{frame}

\begin{frame}
	\frametitle{Bottom-up -- Inverse Resolution}
	Inverse resolution beginnt mit $\mathcal{H} = \{\} = \emptyset$

	\begin{itemize}
		\item $\mathcal{B} = \{b_1,b_2\}$ mit $b_1 = female(mary)$ und
			$b_2 = parent(ann, mary)$
		\item Positives Beispiel $e_1 = daughter(mary, ann)$
	\end{itemize}
\end{frame}

\begin{frame}
	\frametitle{Bottom-up -- Inverse Resolution}
	Algorithmus:
	\begin{itemize}
		\item [1.] Finde Klausel $c_1$, sodass $\{b_2\} \cup c_1 \vDash e_1$
		\item [2.] Finde Klausel $c$\hspace{4pt}, sodass $\{b_1\} \cup c\hspace{4pt} \vDash c_1$
	\end{itemize}
		\begin{figure}[H]
			\begin{center}
				\begin{tikzpicture}[scale=0.8]
					\node (A) at (0.3, 0) {\tiny{$e_1 =daugher(mary,ann)$}};
					\node (B) at (-3.6,2.15) {\tiny{$b_2 = parent(ann,mary)$}};
					\node (C) at (-2,4) {\tiny{$b_1 = female(mary)$}};
					\node (D) at (2.5,5) {\tiny{$c = daughter(X,Y) \leftarrow female(X), parent(Y,X)$}};
					\node (E) at (1.3, 2.2) {\hspace{1cm}\tiny{$c_1 = daughter(mary, Y) \leftarrow parent(Y,mary)$}};
					\node (S1) at (2.3, 1.2) {\tiny{$\theta_2^{-1} = \{ann/Y\}$}};
					\node (S1) at (3, 3.6) {\hspace{0.5cm}\tiny{$\theta_1^{-1} = \{mary/X\}$}};


					\path [->] (B) edge node[above] {} (A);
					\path [->] (C) edge node[above] {} (E);
					\path [<-] (D) edge node[above] {} (E);
					\path [<-] (E) edge node[above] {} (A);
				\end{tikzpicture}
			\end{center}
		\end{figure}
\end{frame}

\begin{frame}
	\frametitle{Bottom-up -- Inverse Resolution}
	Zusammenfassung:
	Problem der \textit{Inverse resolution}:
	\begin{itemize}
		\item Nicht-deterministisch
		\item Hohe Komplexität (großer Suchraum)
	\end{itemize}
	$\Rightarrow$ Aktueller Gegenstand der Forschung
\end{frame}

\subsection{Top-Down -- Refinement-Operator}
\begin{frame}[fragile]
	\frametitle{Top-Down -- Refinement-Operator}
	Idee: Klauseln immer weiter spezifizieren

	\begin{enumerate}
		\item Beginne mit Prädikat von dem Beispiele gebaut sind
		\item Füge solange Literale in Header hinzu, bis passende Beispiele erfüllt sind.
	\end{enumerate}



	\begin{block}{Der Refinement-Operator $\rho$ [Shapiro '83]}
		\begin{align*}
			\forall D \in \rho(C). C \preceq D
		\end{align*}
		Gewünschte Eigenschaften:
		\begin{itemize}
			\item \textit{Finite}: Erhalte endliche Menge an Klauseln
			\item \textit{Complete}: Generierung aller subsummierenden Klauseln
			\item \textit{Non-redundant}: Jede Klausel kann nur auf einen Weg generiert werden
				(Vernachlässigbar)
		\end{itemize}
	\end{block}
\end{frame}

\begin{frame}
\definecolor{wine-red}{rgb}{0.8,0.1,0.6}
	\frametitle{Top-Down -- Refinement-Operator}
	Beispiel für $\rho$:
	\begin{figure}[H]
		\begin{center}
			\begin{tikzpicture}[scale=0.93]
				\node (A) at (1.5,1)   {$p(X,Y)$};
				\node (B) at (-3,-0.5) {\color{blue}$p(X,X)$};
				\node (C) at (-0.6,-1) {\color{red}$p(f(Z_1,Z_2), Y)$};
				\node (D) at (3.4,-1)  {\color{red}$p(X,f(Z_1,Z_2))$};
				\node (E)[text width=3cm] at (7,-0.5) {\color{wine-red}$p(X,Y) \leftarrow q(Z_1, Z_2)$};

				\path [->, blue] (A) edge node[above] {op2} (B);
				\path [->, red]  (A) edge node[right] {op1} (C);
				\path [->, red]  (A) edge node[right] {op1} (D);
				\path [->, wine-red]       (A) edge node[above] {op3} (E);
			\end{tikzpicture}
		\end{center}
		\caption{Beispiel für einen Refinement-operator}
		\label{fig:refinement_operator}
	\end{figure}
\end{frame}

\begin{frame}
	\frametitle{Top-Down -- Refinement-Operator}
	\begin{figure}[H]
		\begin{center}
			\begin{tikzpicture}[scale=1.2]
				\node[font=\tiny] (A) at (1.5,1) {$daughter(X,Y)\leftarrow$};
				\node[font=\tiny] (B) at (-1,0) {$daughter(X,Y) \leftarrow X = Y$};
				\node[font=\tiny] (C) at (0,-1) {$daughter(X,Y) \leftarrow female(X)$};
				\node[font=\tiny] (D) at (2.5,-0.5) {$daughter(X,Y) \leftarrow parent(Y,X)$};
				\node[font=\tiny] (E) at (4.5,0) {$daughter(X,Y) \leftarrow parent(X, Z)$};
				\node[font=\tiny] (F) at (-2,-1.75) {$daughter(X,Y)\leftarrow female(X), female(Y)$};
				\node[font=\tiny] (G) at (2.75, -1.75) {$daughter(X,Y) \leftarrow female(X), parent(Y,X)$};
				\node[font=\tiny] (X) at (1.35, -0) {$\ldots$};
				\node[font=\tiny] (Y) at (0.3, -1.5) {$\ldots$};

				\path [->] (A) edge node[left] {} (B);
				\path [->] (A) edge node[left] {} (C);
				\path [->] (A) edge node[left] {} (D);
				\path [->] (A) edge node[left] {} (E);
				\path [->] (C) edge node[left] {} (F);
				\path [->] (C) edge node[left] {} (G);
			\end{tikzpicture}
		\end{center}
		\caption{Teil des Refinementgraphs für Familienkonstellation}
		\label{fig:refinement_operator}
	\end{figure}
	\begin{block}{Refinement-Operator}
		$\rho(c) = \{daughter(X, Y) \leftarrow L\}$ mit
		\begin{enumerate}
			\item Argumente von $L$ sind Variablen vom Header
			\item $L$ darf eine neue Variable einführen
		\end{enumerate}
	\end{block}
\end{frame}

\begin{frame}
	\frametitle{Top-Down -- Varianten}
	Wie durchsucht man den Refinementgraph?
	\begin{itemize}
		\item Variante 1: Vollständig (\emph{langsam})
		\item Variante 2: Greedy      (\emph{unvollständig})
	\end{itemize}
	\begin{block}{Verbesserung von Top-Down}
		Verkleinere Suchraum durch Seed
	\end{block}
	\begin{center}
		\begin{tikzpicture}
			\node[font=\tiny] (A) at (1.5,1) {$p(X,Y)$};
				\node[font=\tiny] (B) at (-1,0) {$p(X,X)$};
				\node[font=\tiny] (C) at (0.55,0) {$p(X,Y) \leftarrow r(U)$};
				\node[font=\tiny] (D) at (2.6,0) {$p(X,Y) \leftarrow q(V)$};
				\node[font=\tiny] (E) at (1.5,-2) {$p(X,Y) \leftarrow r(X), q(X), q(Y)$};
				\node[font=\tiny] (X) at (1.75, 0.5) {$\ldots$};
				\node[font=\tiny] (Y) at (0.6, 0.5) {$\ldots$};
				\node[font=\tiny] (B2) at (0.3,-1.3)  {};
				\node[font=\tiny] (C2) at (1.5,-1) {};
				\node[font=\tiny] (D2) at (3,-1.3) {};
				\node[font=\tiny] (B3) at (-1,-1.5) {\ldots};
				\node[font=\tiny] (D3) at (4,-1.5) {\ldots};
				\node[font=\tiny] (X2) at (1, -1.25) {$\ldots$};
				\node[font=\tiny] (Y2) at (2, -1.25) {$\ldots$};

				\path [->] (A) edge node[left] {} (B);
				\path [->] (A) edge node[left] {} (C);
				\path [->] (A) edge node[left] {} (D);
				\path [->] (B2) edge node[left] {} (E);
				\path [->] (C2) edge node[left] {} (E);
				\path [->] (D2) edge node[left] {} (E);
			\begin{scope}[label distance=0mm,]
				\coordinate  (aux1) at ([yshift=-15pt]A);
				\coordinate  (aux2) at ([yshift=+10pt]E);
				\node[regular polygon,regular polygon sides=5,draw, red,fit={(aux1) (aux2)},label=right:{\color{red}\small{Bereich mit Seed}}] {};
			\end{scope}
			%\begin{scope}[label distance=6mm,]
			%	\foreach \i/\label in {1/Label 1,2/Label 2,3/Label 3}
			%	{
			%		  \coordinate  (aux\i) at ([xshift=2cm]t\i);
			%		    \node[inner sep=6pt,rounded corners=6pt,draw,red,fit={(t\i) (b\i)
			%			(aux\i)},label=right:{\color{red}\label}] {};
			%			}
			%\end{scope}
		\end{tikzpicture}
	\end{center}
\end{frame}

\section{Anwendungsgebiete}
\subsection{QSAR}
\begin{frame}
	\frametitle{Quantitative structure-activity relationships}
	Problemstellung aus \textbf{Pharmazie}: Stoffe entwickeln mit speziellen biologische Eigenschaften.
	Welche Restgruppenkombination führt zu gewünschten Eigenschaften?\\

	Restgruppen ($OH, F, Br, NH_2, No_2, \ldots)$ haben verschiedene
	chemisch-physikalische Eigenschaften
	\begin{itemize}
		\item Polarität
		\item Größe
		\item Symmetrie
		\item Basisch, sauer
		\item {\hspace{5pt}\vdots}
	\end{itemize}
\end{frame}
\begin{frame}
	\frametitle{Quantitative structure-activity relationships}
	\begin{block}{Beispiel: Thrimethoprim}
		\img{qsar}{Trimethoprim mit Restgruppen}{fig:trim}{0.6}
	\end{block}
	Welche Restgruppenkombination führt zu welchen biologischen Eigenschaften?
\end{frame}
\begin{frame}[fragile]
	\frametitle{Quantitative structure-activity relationships}
	Standard Verfahren nach \textit{Hansch}: Vorhersage durch Bildung von
	Gleichungssystem über phy.-chem. Eigenschaften der Restgruppen

	\begin{block}{ILP-Ansatz}
	Hintergrundwissen erstellen durch bereits bekannte biologische Eigenschaften
	von Stoffen.
	\begin{itemize}
		\item Jeder Restgryppe Eigenschaften zuweisen: $polar(NO_2, polar\_value), \ldots$
		\item $great(DrugX, DrugY)$: $X$ hat höhere biologische Aktivität als $Y$
		\item Regeln für $great(A,B)$ erstellen: 
			\begin{lstlisting}[language=prolog]
				great(A,B) $\leftarrow$ struc(A,D,E,F), struc(B,h,C,h), flex(D,G),
				less_4_flex(G), h_donor(D, h_don_0), pi(D, po_don_1).
			\end{lstlisting}
	\end{itemize}
	\end{block}
\end{frame}

\begin{frame}
	\frametitle{Quantitative structure-activity relationships}
	Bewertung:
	\begin{center}
		\begin{tabular}{|c|c|c|}
			\hline
			& ILP & Hansch\\
			\hline
			Korrelation Training & 0.92 & 0.79\\
			\hline
			Korrelation Test     & 0.46 -- 0.54 & 0.42\\
			\hline
		\end{tabular}
		\captionof{table}{Korelationstabelle für QSAR}
		\end{center}
		Erkenntnis:\\
		ILP funktioniert gut solange Moleküle nicht zu komplex sind.
\end{frame}

\subsection{Weitere}
\begin{frame}
	\frametitle{Zusammenfassung}
	Weitere Anwendungen in Real-World:
	\begin{itemize}
		\item Sekundäre Protein Struktur vorhersagen
		\item Regeln für Rheuma Früherkennung finden
	\end{itemize}
\end{frame}

\section{Zusammenfassung}
\begin{frame}
	\frametitle{Zusammenfassung}
	Weitere Anwendungen in Real-World:
	Bedingungen für ILP:
	\begin{itemize}
		\item \textbf{Formalierbarkeit} von Problemen
		\item Überschaubare Komplexität
		\item Je nach Problem sind ILP-Systeme besser/schlechter geeignet
	\end{itemize}
\end{frame}

\section{Quellen}
\begin{frame}
    \frametitle{Quellen}
    \begin{itemize}
        \item Muggleton, S. \glqq Inductive Logic Programming-Techniques and Applications.\grqq  Chi Chester, UK: Ellis Harwood (1992).
    \end{itemize}
\end{frame}

\begin{frame}[allowframebreaks]{Quellen}
    \raggedright
    \nocite{*}
    \bibliographystyle{plain}
    \bibliography{ref}
\end{frame}

\end{document}
