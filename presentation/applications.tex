\section{Anwendungsbeispiel}
\frame{\frametitle{Agenda}\tableofcontents[currentsection]}
\begin{frame}
	\frametitle{Neulich in der Erlanger Zeitung 23.01.17}
	\img{stinker}{Die Stinker}{fig:stinker}{1.0}
\end{frame}

\begin{frame}
	\frametitle{Quantitative structure-activity relationships}
	Problemstellung aus \textbf{Pharmazie}: Stoffe entwickeln mit speziellen biologische Eigenschaften.
	Welche Restgruppenkombination führt zu gewünschten Eigenschaften?\\

	Restgruppen ($OH, F, Br, NH_2, No_2, \ldots)$ haben verschiedene
	chemisch-physikalische Eigenschaften
	\begin{itemize}
		\item Polarität
		\item Größe
		\item Symmetrie
		\item Basisch, sauer
		\item {\hspace{5pt}\vdots}
	\end{itemize}
\end{frame}
\begin{frame}
	\frametitle{Quantitative structure-activity relationships}
	\begin{block}{Beispiel: Thrimethoprim}
		\img{qsar}{Trimethoprim mit Restgruppen}{fig:trim}{0.6}
	\end{block}
	Welche Restgruppenkombination führt zu welchen biologischen Eigenschaften?
\end{frame}
\begin{frame}[fragile]
	\frametitle{Quantitative structure-activity relationships}
	Standard Verfahren nach \textit{Hansch}: Vorhersage durch Bildung von
	Gleichungssystem über phy.-chem. Eigenschaften der Restgruppen

	\begin{block}{ILP-Ansatz}
	Hintergrundwissen erstellen durch bereits bekannte biologische Eigenschaften
	von Stoffen.
	\begin{itemize}
		\item Jeder Restgruppe Eigenschaften zuweisen: $polar(NO_2, polar\_value), \ldots$
		\item $great(DrugX, DrugY)$: $X$ hat höhere biologische Aktivität als $Y$
		\item Regeln für $great(A,B)$ erstellen: 
			\begin{lstlisting}[language=prolog]
				great(A,B) $\leftarrow$ struc(A,D,E,F), struc(B,h,C,h), flex(D,G),
				less_4_flex(G), h_donor(D, h_don_0), pi(D, po_don_1).
			\end{lstlisting}
	\end{itemize}
	\end{block}
\end{frame}

\begin{frame}
	\frametitle{Quantitative structure-activity relationships}
	Bewertung:
	\begin{center}
		\begin{tabular}{|c|c|c|}
			\hline
			& ILP & Hansch\\
			\hline
			Korrelation Training & 0.92 & 0.79\\
			\hline
			Korrelation Test     & 0.46 -- 0.54 & 0.42\\
			\hline
		\end{tabular}
		\captionof{table}{Korelationstabelle für QSAR}
		\end{center}
		Erkenntnis:\\
		ILP funktioniert gut solange Moleküle nicht zu komplex sind.
\end{frame}

\begin{frame}
	\frametitle{Weitere Real-World Anwendungen}
	Weitere Anwendungen in Real-World:
	\begin{itemize}
		\item Sekundäre Protein Struktur vorhersagen
		\item Regeln für Rheuma Früherkennung finden
	\end{itemize}
\end{frame}
