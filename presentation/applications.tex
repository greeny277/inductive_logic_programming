\section{Anwendungsbeispiel}
\frame{\frametitle{Agenda}\tableofcontents[currentsection]}
\begin{frame}
	\frametitle{Neulich in den Erlanger Nachrichten 23.01.17}
	\img{stinker}{Die Stinker}{fig:stinker}{1.0}
\end{frame}

\begin{frame}
	\frametitle{Quantitative structure-activity relationships}
	Problemstellung aus \textbf{Pharmazie}: Stoffe entwickeln mit speziellen biologische Eigenschaften.
	Welche Restgruppenkombination führt zu gewünschten Eigenschaften?\\

	Restgruppen ($OH, F, Br, NH_2, No_2, \ldots)$ haben verschiedene
	chemisch-physikalische Eigenschaften
	\begin{itemize}
		\item Polarität
		\item Größe
		\item Symmetrie
		\item Basisch, sauer
		\item {\hspace{5pt}\vdots}
	\end{itemize}
\end{frame}
\begin{frame}
	\frametitle{Quantitative structure-activity relationships}
	\begin{block}{Beispiel: Thrimethoprim}
		\img{qsar}{Trimethoprim mit Restgruppen}{fig:trim}{0.6}
	\end{block}
	Welche Restgruppenkombination führt zu welchen biologischen Eigenschaften?
\end{frame}

\begin{frame}
	\frametitle{Quantitative structure-activity relationships}
	\begin{block}{Standard Verfahren nach \textit{Hansch}}
	Vorhersage biologischer Eigenschaften von \emph{unbekannten} Stoffen
	durch Bildung von Gleichungssystem über physikalisch-chemische Eigenschaften der Restgruppen.
	\end{block}
\end{frame}

\begin{frame}
	\frametitle{Quantitative structure-activity relationships}
	\begin{block}{ILP-Ansatz}
		\emph{Idee}: Vergleich des neuen Stoffes mit Eigenschaften von bereits \emph{bekannten}
		Stoffen.\\
	\end{block}
		Hintergrundwissen:
		\begin{enumerate}
			\item { $9$ Prädikate für chemische Eigenschaften der Restgruppen:\\
				$polar(subst, value), size(subst, value),
				h\_donor(subst, value), \ldots$}
			\item { Prädikate für Relation von chemischen Eigenschaften:\\
				$great\_polar(polar\_value1, polar\_value2), \ldots$}
			\item {Struktur von Stoffen:\\
				$struc(Drug, R_3, R_4, R_5)$ }
		\end{enumerate}
\end{frame}

\begin{frame}
	\frametitle{Quantitative structure-activity relationships}
	\begin{block}{Zielrelation}
		\begin{align*}
			great(DrugX, DrugY)
		\end{align*}
	\end{block}
	$\Rightarrow$ Stoff $X$ hat höhere biologische Aktivität als Stoff $Y$.

	\vspace{1cm}

	\textbf{Positive} und \textbf{negative} Beispiele:
	Generiert durch Vergleich biologischer Aktivität von Stoffen. 
	\begin{itemize}
		\item $E^+$: Alle Beispiele für die $great(DrugX, DrugY)$ wahr ist
		\item $E^-$: Alle Beispiele für die $great(DrugX, DrugY)$ falsch ist
	\end{itemize}
\end{frame}

\begin{frame}[fragile]
	\frametitle{Quantitative structure-activity relationships}
	ILP System kann nun Regeln ableiten:
	\begin{bsp}
		\begin{lstlisting}[language=prolog]
			great(A,B) $\leftarrow$ struc(A,D,E,F), struc(B,h,C,h), flex(D,G),
			less_4_flex(G), h_donor(D, h_don_0), pi(D, po_don_1).
		\end{lstlisting}
	\end{bsp}
	$\Rightarrow$ Stoff $A$ hat eine höhere Aktivität, wenn:
	\begin{enumerate}
		\item $B$ keine Restgruppen an Stelle $3$ und $5$
		\item Restgruppe $D$ bei Stoff $A$ spezielle Eigenschaften aufweist
	\end{enumerate}
\end{frame}

\begin{frame}
	\frametitle{Quantitative structure-activity relationships}
	Bewertung:
	\begin{center}
		\begin{tabular}{|c|c|c|}
			\hline
			& ILP & Hansch\\
			\hline
			Korrelation Training & 0.92 & 0.79\\
			\hline
			Korrelation Test     & 0.46 -- 0.54 & 0.42\\
			\hline
		\end{tabular}
		\captionof{table}{Korelationstabelle für QSAR}
	\end{center}
		Erkenntnis:\\
		ILP funktioniert gut solange Moleküle nicht zu komplex sind.
\end{frame}

\begin{frame}
	\frametitle{Weitere Real-World Anwendungen}
	Weitere Anwendungen in Real-World:
	\begin{itemize}
		\item Sekundäre Protein Struktur vorhersagen
		\item Regeln für Rheuma Früherkennung finden
	\end{itemize}
\end{frame}
