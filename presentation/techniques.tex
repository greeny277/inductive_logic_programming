\section{Techniken}

\subsection{Finden von Hypothesenraum}
\begin{frame}
	\frametitle{Suchraum}
	Finden einer neuen Hypothese einfacher durch Eingrenzung von Suchraum

	Einführung der \textbf{Subsumption}
	\begin{block}{Definition: Subsumption für Literale/Atome}
		Seien $L_1$ und $L_2$ Literale: $L_1 \underbrace{\preceq}_{\text{subsummiert}} L_2$
		wenn eine Substitution $\theta$ existert, sodass:
		\begin{align*}
			 L_1 \theta \subseteq L_2
		\end{align*}
		Beispiel:
		\begin{align*}
			 daughter(X, Y)  \subseteq daughter(mary, ann) \text{  mit  } \theta = \{X/mary, Y/ann\}
		\end{align*}
	\end{block}
\end{frame}
\subsection{Finden von Hypothesenraum}
\begin{frame}
	\frametitle{Suchraum}
	Das gleiche Prozedere für Klauseln:
	\begin{block}{Definition: Subsumption für Klauseln}
		Seien $C_1$ und $C_2$ Literale: $C_1 \underbrace{\preceq}_{\text{subsummiert}} C_2$
		wenn eine Substitution $\theta$ existert, sodass:
		\begin{align*}
			 C_1 \theta \subseteq C_2
		\end{align*}
		Beispiel:
		\begin{gather*}
			 daughter(X, Y) \leftarrow parent(Y,X) \subseteq\\
			 daughter(mary, ann) \leftarrow parent(ann, mary), parent(ann, tom)\\
			 \text{  mit  } \theta = \{X/mary, Y/ann\}
		\end{gather*}
	\end{block}
\end{frame}
